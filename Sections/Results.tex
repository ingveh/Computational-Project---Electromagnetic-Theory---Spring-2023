We constructed a discrete coordinate system ranging from $-5R_0$ to $5R_0$ along each coordinate axis divided into $7\cross 7\cross 7$ coordinate points, and calculated the magnetic field of a dipole at the origin in these points using our implementation of eq. \ref{eq:dipoleField}. The resulting field with the corresponding dipole moment is presented in 3D in Figure \ref{fig:dipoleField}, and the planes through the origin are plotted in 2D in Figure \ref{fig:dipolePlanes}\\

\begin{figure}[H]
    \centering
    \includegraphics[scale=.4]{Images/dipoleField.png}
    \caption{Plot of the magnetic field of a pure dipole without magnitude and its corresponding magnetic dipole moment.}
    \label{fig:dipoleField}
\end{figure}

\noindent It can be observed in Figure \ref{fig:dipoleField} that the dipole field has azimuthal symmetry around an axis along the dipole moment and that the field lines point from the magnetic north to the magnetic south, which is exactly how a magnetic dipole behaves. This plot does not portray strength, only direction.\\
\\
We study the field further by assessing the $xz$-plane and $yz$-plane through the origin in shown in Figure \ref{fig:dipolePlanes}. Since the differences in magnetic field strength vary a lot with distance from the source, the strength in Figure \ref{fig:dipolePlanes} has been cut off at $1000B_0$, meaning differences near the center are not discernible.\\

\begin{figure}[H]
    \centering
    \includegraphics[scale=.28]{Images/dipolePlanes.png}
    \caption{Direction and strength of the magnetic field of our dipole in the $xz$-plane (upper images) and in the $yz$-plane (lower images).}
    \label{fig:dipolePlanes}
\end{figure}

\noindent Most apparent is the fact that the magnetic field is tilted in the $yz$-plane, which corresponds to the tilt we imposed on our dipole moment. This tilt can also be seen in Figure \ref{fig:dipoleField}, though it is harder to see. A note about Figure \ref{fig:dipolePlanes} is that the strength of the field, though exaggerated in the plots, has an elliptical shape along the dipole moment. This corresponds to the real behaviour of a dipole.\\
\\
We now turn our attention to the trajectory for a proton travelling through the earth's magnetic field, for which the plots in figures \ref{fig:3dTrajectory12}, \ref{fig:planeTrajectory1} and \ref{fig:3dTrajectory3} correspond to. Each figure has varied parameters.\\

\begin{figure}[H]
    \centering
    \includegraphics[scale=.4]{Images/protonTrajectories12.png}
    \caption{Trajectories of a proton in earth's magnetic field plotted in 3D. The orientation of the tilt is varied in the two curves. Initial conditions $r_i=R_0[-6,0,0]$, $v_i=v_0[10,0,0]$, $\beta=11^{\circ}$}
    \label{fig:3dTrajectory12}
\end{figure}

\noindent There are two key aspects of the trajectory of the particles: they move in a spiralling motion, and they are deflected around the origin. The spiralling motion is seen to occur perpendicular to the magnetic field, which is due to the cross product in eq. \ref{eq:LorentzForce}. The reason for why the particle does not just move in a circle is because the field becomes weaker as the proton moves away for the origin, meaning its spiral motion gets less sharp and the particle translates. The translation happens perpendicular to the field, and also around the origin because the field continuously accelerates the proton in a spiralling motion such that it keeps a consistent distance to the origin, but translates along the field. The fact that the particle moves along the field is apparent in Figure \ref{fig:planeTrajectory1} since the motion in the $yz$-plane is in one plane perpendicular to the dipole moment. The spiralling motion is only present in the $xz$- and the $xy$-planes.

\begin{figure}[H]
    \centering
    \includegraphics[scale=.20]{Images/trajectoryPlanes1.png}
    \caption{Trajectory of a proton in earth's magnetic field in the various planes through the origin. Initial conditions $r_i=R_0[-6,0,0]$, $v_i=v_0[10,0,0]$, $\beta=11^{\circ}$ and $\alpha=0^{\circ}$}
    \label{fig:planeTrajectory1}
\end{figure}

In Figure \ref{fig:3dTrajectory12} we have plotted two different trajectories; identical initial conditions except one has $\alpha=0^{\circ}$ and one has $\alpha=180^{\circ}$. The effect of this is that the plane of motion for the proton is changed in accord to the direction of the dipole moment; the proton must travel in a plane perpendicular to the field.

\begin{figure}[H]
    \centering
    \includegraphics[scale=.4]{Images/protonTrajectory3.png}
    \caption{Trajectory of proton in earth's magnetic field. Initial conditions $r_i=R_0[-6,0,0]$, $v_i=v_0[8,0,0]$, $\beta=11^{\circ}$ and $\alpha=0^{\circ}$}
    \label{fig:3dTrajectory3}
\end{figure}

\noindent The proton in Figure \ref{fig:3dTrajectory3} differs from the others in that the initial velocity is less than those in Figure \ref{fig:3dTrajectory12}. The effect of this lower speed is that the spirals are smaller and that the proton does not travel as far. In Figure \ref{fig:3dTrajectory3} we have in addition to plotting the trajectory also plotted the magnetic field. This is to make it a bit more apparent that the proton moves perpendicular to the field.\\
\\
The last proton trajectory we shall examine is that of a proton which does not start along the $x$-axis. In figures \ref{fig:3dSquiggle} and \ref{fig:squigglyPlane} we see the trajectories of such a proton in 3D and in the $yz$-plane. This trajectory appears much more chaotic, but does have some beautiful symmetries to it. Firstly, the proton is still deflected around the earth. Secondly, the trajectory is still a spiral as seen from above ($xy$-plane). What is interesting though is the fact that the proton seems to oscillate along the direction of the dipole moment. This is the most prominent change that occurs when the proton travels off center.

\begin{figure}[H]
    \centering
    \includegraphics[scale=.4]{Images/squigglyTrajectory.png}
    \caption{Trajectory of proton in earth's magnetic field. Initial conditions $r_i=R_0[-6,0,0]$, $v_i=v_0[8,0,0]$, $\beta=11^{\circ}$ and $\alpha=0^{\circ}$}
    \label{fig:3dSquiggle}
\end{figure}

\begin{figure}[H]
    \centering
    \includegraphics[scale=.4]{Images/squigglyPlane.png}
    \caption{Trajectory of proton in earth's magnetic field in the $yz$-plane. Initial conditions $r_i=R_0[-6,0,0]$, $v_i=v_0[8,0,0]$, $\beta=11^{\circ}$ and $\alpha=0^{\circ}$}
    \label{fig:squigglyPlane}
\end{figure}

\noindent As mentioned in the theory section, the magnetic field does not do any work on the proton, meaning its kinetic energy must remain constant. Change in kinetic energy serves then as an indicator of the accuracy of our simulations. We calculate the relative kinetic energy of a given proton to see how many percent it deviates from the initial kinetic energy. The relative kinetic energy is plotted as a function of time in Figure \ref{fig:kinEnergy3}.

\begin{figure}[H]
    \centering
    \includegraphics[scale=.4]{Images/KinEnergy3.png}
    \caption{Relative kinetic energy of a proton as a function of time. Initial conditions $r_i=R_0[-6,0,0]$, $v_i=v_0[10,0,0]$, $\beta=11^{\circ}$ and $\alpha=0^{\circ}$}
    \label{fig:kinEnergy3}
\end{figure}

\noindent Figure \ref{fig:kinEnergy3} shows that the relative kinetic energy stays approximately constant to within to $0.0000081\%$, which we can view as an acceptable error. Our calculations are with that very accurate.\\
\\
We remark that the most important effect of the magnetic field of the earth is to shield it from phenomena such as solar winds. In our simulations we see just how it happens and in some sense why. The magnetic field continuously changes the direction of the incoming particles without doing any work and deflects them in a spiralling motion. This all happens because of how magnetic fields interact with moving particles; through a cross product and therefore perpendicular to the direction of motion. A theoretical electric field in place of the magnetic field could be thought to be much less effective at shielding, because it would need to do work to decelerate and scatter the incoming particles.\\
\\
Lastly, it is worth mentioning that this indeed is a very simplified model of the northern lights. Actual solar winds consist of large clouds of particles which in turn interact amongst themselves and produce themselves electromagnetic fields. This is also evident in our plots. The spiralling motion of the protons can be thought of as current carrying loops. Such loops produce a magnetic field, and by the right hand rule one can see in our plots that the protons would produce a magnetic field which would oppose the field of the earth. The solar wind would thus change the magnetic field drastically. In reality, the magnetic field of the earth gets dramatically skewed by solar winds, but do in fact shield the earth similarly to what we have seen in or simulations.\\