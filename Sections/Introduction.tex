As my mother and I were taking a stroll that fateful Christmas night, I indeed immediately began my work to simulate the particle trajectories of the northern lights. Thankfully, the embarrassment of not knowing how they looked was to be short lived, because the plots I produced were as beautiful as they were fascinating. Though my mother did not think the trajectories made sense, I, the unseasoned physicist, stood astonished in the woods watching my Jupyter plots while my mother was shivering from the dreadful cold.\\
\\
I will in this report present a model for the earth's magnetic field and numerical calculations of the trajectories for moving particles in this field. This very simplified model will serve to give insight into how particles in solar winds move under the influence of the earth's magnetic field. The analysis consists of testing different initial conditions for the particles, as well as testing different orientations of the earth's magnetic field, as it would vary throughout the year whilst rotating around the sun. Lastly I discuss different aspects of the results, such as how initial position, initial velocity, and orientation of the field change the path of the particles and discuss the accuracy of my results.